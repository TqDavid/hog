\documentclass[times, utf8, seminar, numeric]{fer}
\usepackage{booktabs, url, hyperref}
\usepackage{verbatim}
\usepackage{moreverb}
\usepackage{subfigure}
\usepackage{caption}
\usepackage{epstopdf}
\usepackage{amsthm}

\hypersetup{
   colorlinks,
   citecolor=black,
   filecolor=black,
   linkcolor=black,
   urlcolor=black
}

%dodatak za programski kod
\usepackage{listings}
\usepackage{color}
\usepackage{setspace}
\definecolor{dkgreen}{rgb}{0,0.6,0}
\definecolor{gray}{rgb}{0.5,0.5,0.5}
\definecolor{mauve}{rgb}{0.58,0,0.82}

\lstset{frame=tb,
  language=Java,
  aboveskip=3mm,
  belowskip=3mm,
  showstringspaces=false,
  columns=flexible,
  basicstyle={\small\ttfamily},
  numbers=left,
  numberstyle=\small\color{gray},
  keywordstyle=\color{blue},
  commentstyle=\color{dkgreen},
  stringstyle=\color{mauve},
  breaklines=true,
  breakatwhitespace=true,
  tabsize=2
}


\begin{document}

% TODO: Navedite naslov rada.
\title{Histogram of Oriented Gradients for detection and tracing people}

% TODO: Navedite vaše ime i prezime.
\author{Petra Bevandić \\ Dragan Drandić \\ Melita Kokot \\ Igor Smolkovič \\ Dino Šantl}

\maketitle

\tableofcontents

\chapter{Prikupljena literatura}

\section{HOG članci}

\subsection{Histogram of Oriented Gradients for Human Detection}

\begin{itemize}

\item \textbf{Histogram of Oriented Gradients for Human Detection} 
\item Navneet Dalal, i Bill Triggs
\item U International conference on Computer Vision \& Pattern Recognition, Vol. 1, stranice 886-893, Lipanj 2005.
	\begin{itemize} 
		\item URL: \url{http://lear.inrialpes.fr/people/triggs/pubs/Dalal-cvpr05.pdf}
		\item opisan histogram orijentiranih gradijenata
		\item analiziran utjecaj raznih parametara histograma na performanse
		\item HOG primjenjene na prepoznavanje pješaka (uz korištenje SVM-a)
		\item MIT i Inria baza pješaka, cca 2500 slika, originalne reflektirane
		\item izdvajanje "teških" primjera
		\item postojanje okoline oko osobe
	\end{itemize}
\end{itemize}

\subsection{Detection Using a CAscade of Histograms of Oriented Gradients}
\begin{itemize}
\item \textbf{Detection Using a CAscade of Histograms of Oriented Gradients}
\item Qiang Zhu, Shai Avidan, Mei-Chen Yeh, i Kwang-Ting Cheng. Fast Human 
\item U International conference on Computer Vision \& Pattern Recognition, Vol. 2, stranice 1491-1498, Lipanj 2006. 
	\begin{itemize}
		\item URL: \url{http://citeseerx.ist.psu.edu/viewdoc/download?doi=10.1.1.68.6232&rep=rep1&type=pdf}
		\item ubrzanje traženja ljudi u slikama
		\item ne koriste blokove fiksne veličine - globalne karakteristike
		\item Ada-Boost
		\item korištenje "integralne slike" - koriste se HOG-ovi izračunati za manje blokove da bi se ubrzalo računanje histograma za veće blokove
		\item slučajan odabir blokova koji idu u klasifikator
	\end{itemize}
\end{itemize}

\subsection{Pedestrian Detection Using Infrared Images and Histograms of Oriented Gradients}
\begin{itemize}
\item \textbf{Pedestrian Detection Using Infrared Images and Histograms of Oriented Gradients}
\item Frédéric Suard, Alain Rakotomamonjy, Abdelaziz Bensrhair, i Alberto Broggi
\item U Intelligent Vehicles Symposium, stranice 206-212, 2006. 
\begin{itemize}
	\item URL: \url{http://citeseerx.ist.psu.edu/viewdoc/download?doi=10.1.1.80.1379&rep=rep1&type=pdf}
	\item detekcija ljudi pomoću infracrvenih slika
	\item HOG značajke, SVM klasifikator
	\item primjena u detekciji pješaka noću
	\item odabir optimalnih parametara svih stavaka sustava za detekciju osoba
	\item osrednji rezultati
\end{itemize}
\end{itemize}

\subsection{Visual Classification of Coarse Vehicle Orientation Using Histogram of Oriented Gradients Features}
\begin{itemize}
\item \textbf{Visual Classification of Coarse Vehicle Orientation Using Histogram of Oriented Gradients Features} 
\item Paul E. Rybski, Daniel Huber, Daniel D. Morris, i Regis Hoffman
\begin{itemize}
		\item URL: \url{http://citeseerx.ist.psu.edu/viewdoc/download?doi=10.1.1.183.4072&rep=rep1&type=pdf}
		\item detekcija orijentacije vozila iz slike (bez informacije o smjeru kretanja)
		\item jednostavna primjena prethodno opisanog algoritma na prepoznavanje vozila
		\item korištena javno dostupna implementacija HOG-a
		\item vrlo dobri rezultati
	\end{itemize}
\end{itemize}

\subsection{Enhancing Real-time Human Detection based on Histograms of Oriented Gradients}
\begin{itemize}
		\item \textbf{Enhancing Real-time Human Detection based on Histograms of Oriented Gradients} 
	\item Marco Pedersoli, Jordi Gonzàlez, Bhaskar Chakraborty, i Juan J. Villanueva
	\begin{itemize} 
		\item URL: \url{http://iselab.cvc.uab.es/files/Publications/2007/PDF/CORES07_MP.pdf}
		\item ubrzanje računanje HOG-a, izrada frameworka za detekciju osoba
		\item (pristupi detekciji osoba - detekcija cijele osobe vs. detekcija dijelova koji su povezani)
		\item AdaBoost
		\item računanje značajki za veće blokove preko značajki za manje blokove
		\item MIT baza osoba
		\end{itemize}
\end{itemize}

\subsection{Integral Histogram: A fast way to Extract Histograms in Cartesian Spaces}
\begin{itemize}
\item \textbf{Integral Histogram: A fast way to Extract Histograms in Cartesian Spaces}
\item Fatih Porikli
\item U International conference on Computer Vision \& Pattern Recognition, Vol. 1, stranice 829-836, Lipanj 2005. 
	\begin{itemize}
		\item URL: \url{http://www.merl.com/reports/docs/TR2005-057.pdf}
		\item primjenjiv u sustavima koji rade u realnom vremenu
		\item koristi se ta bilo koju metodu koja traži maksimalno preklapanje histograma
	\end{itemize}
\end{itemize}

\section{Praćenje objekata}

\begin{itemize}

\item \textbf{Multitarget Tracking of Pedestrians in Video Sequences Based on Particle Filters}, Hui Li, Shengwu Xiong, Pengfei Duan, Xiangzhen Kong

\item \textbf{Object Tracking in Crowded Video Scenes Based on the Undecimated Wavelet Features and Texture Analysis}, M. Khansari, H. R. Rabiee, M. Asadi, M. Ghanbari

\item \textbf{Real-time object detection and tracking for industrial applications}, Selim Benhimane, Hesam Najafi, Matthias Grundmann, Ezio Malis, Yakup Genc, Nassir Navab

\item \textbf{Efficient Tracking of Many Objects in Structured Environments}, Nathan Jacobs, Michael Dixon, Scott Satkin, Robert Pless

\item \textbf{Fast and accurate moving object exraction technique for MPEG-4 object-based video coding}, Ju Guo, Jongwon Kim, C.C: Jay Kuo

\end{itemize}

\chapter{Dijagram planiranog sustava}

\chapter{Baza slika/videa}

\chapter{Programski alati}

%\bibliography{literatura}
%\bibliographystyle{fer} %promijena za citiranje po redu ieeetr
%\bibliographystyle{ieeetr}

\begin{comment}
\begin{sazetak}
Simbolička regresija je postupak otkrivanja matematičkog izraza u skupu podataka. Daje se pregled metoda za simboličku regresiju s naglaskom na genetsko programiranje. Obrađuju se problemi kao što su domene funkcija (nisu definirane na cijelom skupu realnih brojeva). Problemi se rješavaju intervalnom aritmetikom i linearnim skaliranjem. Na kraju se ukratko opisuje mogućnost paralelizacije i primjene. 

\kljucnerijeci{genetsko programiranje, s}

\end{sazetak}

% TODO: Navedite naslov na engleskom jeziku.
\engtitle{Application of graphics coprocessors for program execution on stream programming model}

\begin{abstract}


\keywords{GPU, StreamIt, Sponge, StreamGate, CUDA, stream model, filter, optimization, graphics card}
\end{abstract}
\end{comment}

\end{document}
